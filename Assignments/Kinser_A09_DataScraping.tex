% Options for packages loaded elsewhere
\PassOptionsToPackage{unicode}{hyperref}
\PassOptionsToPackage{hyphens}{url}
%
\documentclass[
]{article}
\usepackage{amsmath,amssymb}
\usepackage{lmodern}
\usepackage{iftex}
\ifPDFTeX
  \usepackage[T1]{fontenc}
  \usepackage[utf8]{inputenc}
  \usepackage{textcomp} % provide euro and other symbols
\else % if luatex or xetex
  \usepackage{unicode-math}
  \defaultfontfeatures{Scale=MatchLowercase}
  \defaultfontfeatures[\rmfamily]{Ligatures=TeX,Scale=1}
\fi
% Use upquote if available, for straight quotes in verbatim environments
\IfFileExists{upquote.sty}{\usepackage{upquote}}{}
\IfFileExists{microtype.sty}{% use microtype if available
  \usepackage[]{microtype}
  \UseMicrotypeSet[protrusion]{basicmath} % disable protrusion for tt fonts
}{}
\makeatletter
\@ifundefined{KOMAClassName}{% if non-KOMA class
  \IfFileExists{parskip.sty}{%
    \usepackage{parskip}
  }{% else
    \setlength{\parindent}{0pt}
    \setlength{\parskip}{6pt plus 2pt minus 1pt}}
}{% if KOMA class
  \KOMAoptions{parskip=half}}
\makeatother
\usepackage{xcolor}
\IfFileExists{xurl.sty}{\usepackage{xurl}}{} % add URL line breaks if available
\IfFileExists{bookmark.sty}{\usepackage{bookmark}}{\usepackage{hyperref}}
\hypersetup{
  pdftitle={Assignment 09: Data Scraping},
  pdfauthor={Stephanie Kinser},
  hidelinks,
  pdfcreator={LaTeX via pandoc}}
\urlstyle{same} % disable monospaced font for URLs
\usepackage[margin=2.54cm]{geometry}
\usepackage{color}
\usepackage{fancyvrb}
\newcommand{\VerbBar}{|}
\newcommand{\VERB}{\Verb[commandchars=\\\{\}]}
\DefineVerbatimEnvironment{Highlighting}{Verbatim}{commandchars=\\\{\}}
% Add ',fontsize=\small' for more characters per line
\usepackage{framed}
\definecolor{shadecolor}{RGB}{248,248,248}
\newenvironment{Shaded}{\begin{snugshade}}{\end{snugshade}}
\newcommand{\AlertTok}[1]{\textcolor[rgb]{0.94,0.16,0.16}{#1}}
\newcommand{\AnnotationTok}[1]{\textcolor[rgb]{0.56,0.35,0.01}{\textbf{\textit{#1}}}}
\newcommand{\AttributeTok}[1]{\textcolor[rgb]{0.77,0.63,0.00}{#1}}
\newcommand{\BaseNTok}[1]{\textcolor[rgb]{0.00,0.00,0.81}{#1}}
\newcommand{\BuiltInTok}[1]{#1}
\newcommand{\CharTok}[1]{\textcolor[rgb]{0.31,0.60,0.02}{#1}}
\newcommand{\CommentTok}[1]{\textcolor[rgb]{0.56,0.35,0.01}{\textit{#1}}}
\newcommand{\CommentVarTok}[1]{\textcolor[rgb]{0.56,0.35,0.01}{\textbf{\textit{#1}}}}
\newcommand{\ConstantTok}[1]{\textcolor[rgb]{0.00,0.00,0.00}{#1}}
\newcommand{\ControlFlowTok}[1]{\textcolor[rgb]{0.13,0.29,0.53}{\textbf{#1}}}
\newcommand{\DataTypeTok}[1]{\textcolor[rgb]{0.13,0.29,0.53}{#1}}
\newcommand{\DecValTok}[1]{\textcolor[rgb]{0.00,0.00,0.81}{#1}}
\newcommand{\DocumentationTok}[1]{\textcolor[rgb]{0.56,0.35,0.01}{\textbf{\textit{#1}}}}
\newcommand{\ErrorTok}[1]{\textcolor[rgb]{0.64,0.00,0.00}{\textbf{#1}}}
\newcommand{\ExtensionTok}[1]{#1}
\newcommand{\FloatTok}[1]{\textcolor[rgb]{0.00,0.00,0.81}{#1}}
\newcommand{\FunctionTok}[1]{\textcolor[rgb]{0.00,0.00,0.00}{#1}}
\newcommand{\ImportTok}[1]{#1}
\newcommand{\InformationTok}[1]{\textcolor[rgb]{0.56,0.35,0.01}{\textbf{\textit{#1}}}}
\newcommand{\KeywordTok}[1]{\textcolor[rgb]{0.13,0.29,0.53}{\textbf{#1}}}
\newcommand{\NormalTok}[1]{#1}
\newcommand{\OperatorTok}[1]{\textcolor[rgb]{0.81,0.36,0.00}{\textbf{#1}}}
\newcommand{\OtherTok}[1]{\textcolor[rgb]{0.56,0.35,0.01}{#1}}
\newcommand{\PreprocessorTok}[1]{\textcolor[rgb]{0.56,0.35,0.01}{\textit{#1}}}
\newcommand{\RegionMarkerTok}[1]{#1}
\newcommand{\SpecialCharTok}[1]{\textcolor[rgb]{0.00,0.00,0.00}{#1}}
\newcommand{\SpecialStringTok}[1]{\textcolor[rgb]{0.31,0.60,0.02}{#1}}
\newcommand{\StringTok}[1]{\textcolor[rgb]{0.31,0.60,0.02}{#1}}
\newcommand{\VariableTok}[1]{\textcolor[rgb]{0.00,0.00,0.00}{#1}}
\newcommand{\VerbatimStringTok}[1]{\textcolor[rgb]{0.31,0.60,0.02}{#1}}
\newcommand{\WarningTok}[1]{\textcolor[rgb]{0.56,0.35,0.01}{\textbf{\textit{#1}}}}
\usepackage{graphicx}
\makeatletter
\def\maxwidth{\ifdim\Gin@nat@width>\linewidth\linewidth\else\Gin@nat@width\fi}
\def\maxheight{\ifdim\Gin@nat@height>\textheight\textheight\else\Gin@nat@height\fi}
\makeatother
% Scale images if necessary, so that they will not overflow the page
% margins by default, and it is still possible to overwrite the defaults
% using explicit options in \includegraphics[width, height, ...]{}
\setkeys{Gin}{width=\maxwidth,height=\maxheight,keepaspectratio}
% Set default figure placement to htbp
\makeatletter
\def\fps@figure{htbp}
\makeatother
\setlength{\emergencystretch}{3em} % prevent overfull lines
\providecommand{\tightlist}{%
  \setlength{\itemsep}{0pt}\setlength{\parskip}{0pt}}
\setcounter{secnumdepth}{-\maxdimen} % remove section numbering
\ifLuaTeX
  \usepackage{selnolig}  % disable illegal ligatures
\fi

\title{Assignment 09: Data Scraping}
\author{Stephanie Kinser}
\date{}

\begin{document}
\maketitle

\hypertarget{total-points}{%
\section{Total points:}\label{total-points}}

\hypertarget{overview}{%
\subsection{OVERVIEW}\label{overview}}

This exercise accompanies the lessons in Environmental Data Analytics on
data scraping.

\hypertarget{directions}{%
\subsection{Directions}\label{directions}}

\begin{enumerate}
\def\labelenumi{\arabic{enumi}.}
\tightlist
\item
  Change ``Student Name'' on line 3 (above) with your name.
\item
  Work through the steps, \textbf{creating code and output} that fulfill
  each instruction.
\item
  Be sure to \textbf{answer the questions} in this assignment document.
\item
  When you have completed the assignment, \textbf{Knit} the text and
  code into a single PDF file.
\item
  After Knitting, submit the completed exercise (PDF file) to the
  dropbox in Sakai. Add your last name into the file name (e.g.,
  ``Fay\_09\_Data\_Scraping.Rmd'') prior to submission.
\end{enumerate}

\hypertarget{set-up}{%
\subsection{Set up}\label{set-up}}

\begin{enumerate}
\def\labelenumi{\arabic{enumi}.}
\tightlist
\item
  Set up your session:
\end{enumerate}

\begin{itemize}
\tightlist
\item
  Check your working directory
\item
  Load the packages \texttt{tidyverse}, \texttt{rvest}, and any others
  you end up using.
\item
  Set your ggplot theme
\end{itemize}

\begin{Shaded}
\begin{Highlighting}[]
\CommentTok{\#1}

\FunctionTok{getwd}\NormalTok{()}
\end{Highlighting}
\end{Shaded}

\begin{verbatim}
## [1] "C:/Users/pogo/Documents/ENV872/Environmental_Data_Analytics_2022/Assignments"
\end{verbatim}

\begin{Shaded}
\begin{Highlighting}[]
\FunctionTok{library}\NormalTok{(tidyverse)}
\FunctionTok{library}\NormalTok{(dplyr)}
\CommentTok{\#install.packages("lubridate")}
\FunctionTok{library}\NormalTok{(lubridate)}
\CommentTok{\#install.packages("rvest")}
\FunctionTok{library}\NormalTok{(rvest)}
\FunctionTok{options}\NormalTok{(}\AttributeTok{tinytex.verbose =} \ConstantTok{TRUE}\NormalTok{)}

\NormalTok{A9\_theme }\OtherTok{\textless{}{-}} \FunctionTok{theme\_light}\NormalTok{(}\AttributeTok{base\_size =} \DecValTok{12}\NormalTok{)}\SpecialCharTok{+}
   \FunctionTok{theme}\NormalTok{(}\AttributeTok{axis.text =} \FunctionTok{element\_text}\NormalTok{(}\AttributeTok{color =} \StringTok{"black"}\NormalTok{), }
        \AttributeTok{legend.position =} \StringTok{"right"}\NormalTok{, }\AttributeTok{panel.grid.minor =} \FunctionTok{element\_blank}\NormalTok{())}

\FunctionTok{theme\_set}\NormalTok{(A9\_theme)}
\end{Highlighting}
\end{Shaded}

\begin{enumerate}
\def\labelenumi{\arabic{enumi}.}
\setcounter{enumi}{1}
\tightlist
\item
  We will be scraping data from the NC DEQs Local Water Supply Planning
  website, specifically the Durham's 2019 Municipal Local Water Supply
  Plan (LWSP):
\end{enumerate}

\begin{itemize}
\tightlist
\item
  Navigate to \url{https://www.ncwater.org/WUDC/app/LWSP/search.php}
\item
  Change the date from 2021 to 2020 in the upper right corner.
\item
  Scroll down and select the LWSP link next to Durham Municipality.
\item
  Note the web address:
  \url{https://www.ncwater.org/WUDC/app/LWSP/report.php?pwsid=03-32-010\&year=2020}
\end{itemize}

Indicate this website as the as the URL to be scraped. (In other words,
read the contents into an \texttt{rvest} webpage object.)

\begin{Shaded}
\begin{Highlighting}[]
\CommentTok{\#2}
\NormalTok{webpage }\OtherTok{\textless{}{-}} \FunctionTok{read\_html}\NormalTok{(}\StringTok{\textquotesingle{}https://www.ncwater.org/WUDC/app/LWSP/report.php?pwsid=03{-}32{-}010\&year=2020\textquotesingle{}}\NormalTok{)}

\NormalTok{webpage}
\end{Highlighting}
\end{Shaded}

\begin{verbatim}
## {html_document}
## <html xmlns="http://www.w3.org/1999/xhtml" lang="en" xml:lang="en">
## [1] <head>\n<title>DWR :: Local Water Supply Planning</title>\n<meta http-equ ...
## [2] <body id="plan">\r\n<!--<div id="division-header">\r\n<a name="top" href= ...
\end{verbatim}

\begin{enumerate}
\def\labelenumi{\arabic{enumi}.}
\setcounter{enumi}{2}
\tightlist
\item
  The data we want to collect are listed below:
\end{enumerate}

\begin{itemize}
\item
  From the ``1. System Information'' section:
\item
  Water system name
\item
  PSWID
\item
  Ownership
\item
  From the ``3. Water Supply Sources'' section:
\item
  Average Daily Use (MGD) - for each month
\end{itemize}

In the code chunk below scrape these values, assigning them to three
separate variables.

\begin{quote}
HINT: The first value should be ``Durham'', the second ``03-32-010'',
the third ``Municipality'', and the last should be a vector of 12
numeric values, with the first value being 36.0100.
\end{quote}

\begin{Shaded}
\begin{Highlighting}[]
\CommentTok{\#3}
\NormalTok{water.system.name }\OtherTok{\textless{}{-}}\NormalTok{ webpage }\SpecialCharTok{\%\textgreater{}\%} \FunctionTok{html\_nodes}\NormalTok{(}\StringTok{\textquotesingle{}div+ table tr:nth{-}child(1) td:nth{-}child(2)\textquotesingle{}}\NormalTok{) }\SpecialCharTok{\%\textgreater{}\%} \FunctionTok{html\_text}\NormalTok{()}

\NormalTok{pwsid }\OtherTok{\textless{}{-}}\NormalTok{ webpage }\SpecialCharTok{\%\textgreater{}\%} \FunctionTok{html\_nodes}\NormalTok{(}\StringTok{\textquotesingle{}td tr:nth{-}child(1) td:nth{-}child(5)\textquotesingle{}}\NormalTok{) }\SpecialCharTok{\%\textgreater{}\%} \FunctionTok{html\_text}\NormalTok{()}

\NormalTok{ownership }\OtherTok{\textless{}{-}}\NormalTok{ webpage }\SpecialCharTok{\%\textgreater{}\%} \FunctionTok{html\_nodes}\NormalTok{(}\StringTok{\textquotesingle{}div+ table tr:nth{-}child(2) td:nth{-}child(4)\textquotesingle{}}\NormalTok{) }\SpecialCharTok{\%\textgreater{}\%} \FunctionTok{html\_text}\NormalTok{()}

\NormalTok{max.withdrawals.mgd }\OtherTok{\textless{}{-}}\NormalTok{ webpage }\SpecialCharTok{\%\textgreater{}\%} \FunctionTok{html\_nodes}\NormalTok{(}\StringTok{\textquotesingle{}th\textasciitilde{} td+ td\textquotesingle{}}\NormalTok{) }\SpecialCharTok{\%\textgreater{}\%} \FunctionTok{html\_text}\NormalTok{()}
\end{Highlighting}
\end{Shaded}

\begin{enumerate}
\def\labelenumi{\arabic{enumi}.}
\setcounter{enumi}{3}
\tightlist
\item
  Convert your scraped data into a dataframe. This dataframe should have
  a column for each of the 4 variables scraped and a row for the month
  corresponding to the withdrawal data. Also add a Date column that
  includes your month and year in date format. (Feel free to add a Year
  column too, if you wish.)
\end{enumerate}

\begin{quote}
TIP: Use \texttt{rep()} to repeat a value when creating a dataframe.
\end{quote}

\begin{quote}
NOTE: It's likely you won't be able to scrape the monthly widthrawal
data in order. You can overcome this by creating a month column in the
same order the data are scraped: Jan, May, Sept, Feb, etc\ldots{}
\end{quote}

\begin{enumerate}
\def\labelenumi{\arabic{enumi}.}
\setcounter{enumi}{4}
\tightlist
\item
  Plot the max daily withdrawals across the months for 2020
\end{enumerate}

\begin{Shaded}
\begin{Highlighting}[]
\CommentTok{\#4}
\NormalTok{Month }\OtherTok{\textless{}{-}} \FunctionTok{c}\NormalTok{(}\DecValTok{01}\NormalTok{, }\DecValTok{05}\NormalTok{, }\DecValTok{09}\NormalTok{, }\DecValTok{02}\NormalTok{, }\DecValTok{06}\NormalTok{, }\DecValTok{10}\NormalTok{, }\DecValTok{03}\NormalTok{, }\DecValTok{07}\NormalTok{, }\DecValTok{11}\NormalTok{, }\DecValTok{04}\NormalTok{, }\DecValTok{08}\NormalTok{, }\DecValTok{12}\NormalTok{)}
\NormalTok{Year }\OtherTok{\textless{}{-}} \FunctionTok{rep}\NormalTok{(}\DecValTok{2020}\NormalTok{, }\DecValTok{12}\NormalTok{)}

\NormalTok{water\_df }\OtherTok{\textless{}{-}} \FunctionTok{data.frame}\NormalTok{(}
  \StringTok{"Month"} \OtherTok{=}\NormalTok{ Month,}
  \StringTok{"Year"} \OtherTok{=}\NormalTok{ Year,}
  \StringTok{"Date"} \OtherTok{=} \FunctionTok{my}\NormalTok{(}\FunctionTok{paste}\NormalTok{(Month,}\StringTok{"{-}"}\NormalTok{,Year)),}
  \StringTok{"Water\_System"} \OtherTok{=} \FunctionTok{rep}\NormalTok{(water.system.name, }\DecValTok{12}\NormalTok{),}
  \StringTok{"Owner"} \OtherTok{=} \FunctionTok{rep}\NormalTok{(ownership, }\DecValTok{12}\NormalTok{),}
  \StringTok{"PWSID"} \OtherTok{=} \FunctionTok{rep}\NormalTok{(pwsid, }\DecValTok{12}\NormalTok{),}
  \StringTok{"Max\_Withdrawals"} \OtherTok{=} \FunctionTok{round}\NormalTok{(}\FunctionTok{as.numeric}\NormalTok{(max.withdrawals.mgd), }\DecValTok{2}\NormalTok{)) }\SpecialCharTok{\%\textgreater{}\%}
  \FunctionTok{arrange}\NormalTok{(Date)}


\CommentTok{\#5}
\FunctionTok{ggplot}\NormalTok{(water\_df, }\FunctionTok{aes}\NormalTok{(}\AttributeTok{x =}\NormalTok{ Date, }\AttributeTok{y =}\NormalTok{ Max\_Withdrawals)) }\SpecialCharTok{+}
  \FunctionTok{geom\_point}\NormalTok{() }\SpecialCharTok{+}
  \FunctionTok{labs}\NormalTok{(}\AttributeTok{title =} \StringTok{"Durham Max Water Withdrawals (2020)"}\NormalTok{, }\AttributeTok{x =} \StringTok{"Date"}\NormalTok{, }\AttributeTok{y =} \StringTok{"Maximum Daily Water Withdrawal (MGD)"}\NormalTok{)}
\end{Highlighting}
\end{Shaded}

\includegraphics{Kinser_A09_DataScraping_files/figure-latex/create.a.dataframe.from.scraped.data-1.pdf}

\begin{enumerate}
\def\labelenumi{\arabic{enumi}.}
\setcounter{enumi}{5}
\tightlist
\item
  Note that the PWSID and the year appear in the web address for the
  page we scraped. Construct a function using your code above that can
  scrape data for any PWSID and year for which the NC DEQ has data.
  \textbf{Be sure to modify the code to reflect the year and site
  scraped}.
\end{enumerate}

\begin{Shaded}
\begin{Highlighting}[]
\CommentTok{\#6.}
\NormalTok{scrape.it }\OtherTok{\textless{}{-}} \ControlFlowTok{function}\NormalTok{(PWSID, the\_year)\{}
  
  \CommentTok{\#Get the proper webpage}
\NormalTok{    webpage }\OtherTok{\textless{}{-}} \FunctionTok{read\_html}\NormalTok{(}\FunctionTok{paste0}\NormalTok{(}\StringTok{\textquotesingle{}https://www.ncwater.org/WUDC/app/LWSP/report.php?pwsid=\textquotesingle{}}\NormalTok{, PWSID, }\StringTok{\textquotesingle{}\&year=\textquotesingle{}}\NormalTok{, the\_year))}
    

  \CommentTok{\#Locate elements and read their text attributes into variables}
\NormalTok{  water.system.name }\OtherTok{\textless{}{-}}\NormalTok{ webpage }\SpecialCharTok{\%\textgreater{}\%} \FunctionTok{html\_nodes}\NormalTok{(}\StringTok{\textquotesingle{}div+ table tr:nth{-}child(1) td:nth{-}child(2)\textquotesingle{}}\NormalTok{) }\SpecialCharTok{\%\textgreater{}\%} \FunctionTok{html\_text}\NormalTok{()}

\NormalTok{  pwsid }\OtherTok{\textless{}{-}}\NormalTok{ webpage }\SpecialCharTok{\%\textgreater{}\%} \FunctionTok{html\_nodes}\NormalTok{(}\StringTok{\textquotesingle{}td tr:nth{-}child(1) td:nth{-}child(5)\textquotesingle{}}\NormalTok{) }\SpecialCharTok{\%\textgreater{}\%} \FunctionTok{html\_text}\NormalTok{()}

\NormalTok{  ownership }\OtherTok{\textless{}{-}}\NormalTok{ webpage }\SpecialCharTok{\%\textgreater{}\%} \FunctionTok{html\_nodes}\NormalTok{(}\StringTok{\textquotesingle{}div+ table tr:nth{-}child(2) td:nth{-}child(4)\textquotesingle{}}\NormalTok{) }\SpecialCharTok{\%\textgreater{}\%} \FunctionTok{html\_text}\NormalTok{()}

\NormalTok{  max.withdrawals.mgd }\OtherTok{\textless{}{-}}\NormalTok{ webpage }\SpecialCharTok{\%\textgreater{}\%} \FunctionTok{html\_nodes}\NormalTok{(}\StringTok{\textquotesingle{}th\textasciitilde{} td+ td\textquotesingle{}}\NormalTok{) }\SpecialCharTok{\%\textgreater{}\%} \FunctionTok{html\_text}\NormalTok{()}

  
  \CommentTok{\#Construct a dataframe from the values}
\NormalTok{  Month }\OtherTok{\textless{}{-}} \FunctionTok{c}\NormalTok{(}\DecValTok{01}\NormalTok{, }\DecValTok{05}\NormalTok{, }\DecValTok{09}\NormalTok{, }\DecValTok{02}\NormalTok{, }\DecValTok{06}\NormalTok{, }\DecValTok{10}\NormalTok{, }\DecValTok{03}\NormalTok{, }\DecValTok{07}\NormalTok{, }\DecValTok{11}\NormalTok{, }\DecValTok{04}\NormalTok{, }\DecValTok{08}\NormalTok{, }\DecValTok{12}\NormalTok{)}
\NormalTok{  Year }\OtherTok{\textless{}{-}} \FunctionTok{rep}\NormalTok{(the\_year, }\DecValTok{12}\NormalTok{)}

\NormalTok{  the\_df }\OtherTok{\textless{}{-}} \FunctionTok{data.frame}\NormalTok{(}
    \StringTok{"Month"} \OtherTok{=}\NormalTok{ Month,}
    \StringTok{"Year"} \OtherTok{=}\NormalTok{ Year,}
    \StringTok{"Date"} \OtherTok{=} \FunctionTok{my}\NormalTok{(}\FunctionTok{paste}\NormalTok{(Month,}\StringTok{"{-}"}\NormalTok{,Year)),}
    \StringTok{"Water\_System"} \OtherTok{=} \FunctionTok{rep}\NormalTok{(water.system.name, }\DecValTok{12}\NormalTok{),}
    \StringTok{"Owner"} \OtherTok{=} \FunctionTok{rep}\NormalTok{(ownership, }\DecValTok{12}\NormalTok{),}
    \StringTok{"PWSID"} \OtherTok{=} \FunctionTok{rep}\NormalTok{(pwsid, }\DecValTok{12}\NormalTok{),}
    \StringTok{"Max\_Withdrawals"} \OtherTok{=} \FunctionTok{round}\NormalTok{(}\FunctionTok{as.numeric}\NormalTok{(max.withdrawals.mgd), }\DecValTok{2}\NormalTok{)) }\SpecialCharTok{\%\textgreater{}\%}
    \FunctionTok{arrange}\NormalTok{(Date)}
  
  \FunctionTok{return}\NormalTok{(the\_df)}
\NormalTok{\}}
\end{Highlighting}
\end{Shaded}

\begin{enumerate}
\def\labelenumi{\arabic{enumi}.}
\setcounter{enumi}{6}
\tightlist
\item
  Use the function above to extract and plot max daily withdrawals for
  Durham (PWSID=`03-32-010') for each month in 2015
\end{enumerate}

\begin{Shaded}
\begin{Highlighting}[]
\CommentTok{\#7}
\NormalTok{Durham\_2015 }\OtherTok{\textless{}{-}} \FunctionTok{scrape.it}\NormalTok{(}\StringTok{"03{-}32{-}010"}\NormalTok{, }\DecValTok{2015}\NormalTok{)}

\FunctionTok{ggplot}\NormalTok{(Durham\_2015, }\FunctionTok{aes}\NormalTok{(}\AttributeTok{x =}\NormalTok{ Date, }\AttributeTok{y =}\NormalTok{ Max\_Withdrawals)) }\SpecialCharTok{+}
  \FunctionTok{geom\_point}\NormalTok{() }\SpecialCharTok{+}
  \FunctionTok{labs}\NormalTok{(}\AttributeTok{title =} \StringTok{"Durham Max Water Withdrawals (2015)"}\NormalTok{, }\AttributeTok{x =} \StringTok{"Date"}\NormalTok{, }\AttributeTok{y =} \StringTok{"Maximum Daily Water Withdrawal (MGD)"}\NormalTok{)}
\end{Highlighting}
\end{Shaded}

\includegraphics{Kinser_A09_DataScraping_files/figure-latex/fetch.and.plot.Durham.2015.data-1.pdf}

\begin{enumerate}
\def\labelenumi{\arabic{enumi}.}
\setcounter{enumi}{7}
\tightlist
\item
  Use the function above to extract data for Asheville (PWSID =
  01-11-010) in 2015. Combine this data with the Durham data collected
  above and create a plot that compares the Asheville to Durham's water
  withdrawals.
\end{enumerate}

\begin{Shaded}
\begin{Highlighting}[]
\CommentTok{\#8}
\NormalTok{Asheville\_2015 }\OtherTok{\textless{}{-}} \FunctionTok{scrape.it}\NormalTok{(}\StringTok{"01{-}11{-}010"}\NormalTok{, }\DecValTok{2015}\NormalTok{)}

\NormalTok{water\_2015 }\OtherTok{\textless{}{-}} \FunctionTok{rbind}\NormalTok{(Durham\_2015, Asheville\_2015) }\SpecialCharTok{\%\textgreater{}\%}
  \FunctionTok{group\_by}\NormalTok{(Water\_System)}

\NormalTok{water\_plot }\OtherTok{\textless{}{-}} \FunctionTok{ggplot}\NormalTok{(water\_2015)}\SpecialCharTok{+}
  \FunctionTok{geom\_point}\NormalTok{(}\FunctionTok{aes}\NormalTok{(}\AttributeTok{x =}\NormalTok{ Date, }\AttributeTok{y =}\NormalTok{ Max\_Withdrawals, }\AttributeTok{color =}\NormalTok{ Water\_System))}\SpecialCharTok{+}
  \FunctionTok{labs}\NormalTok{(}\AttributeTok{title =} \StringTok{"Monthly Water Use"}\NormalTok{, }\AttributeTok{x =} \StringTok{"Month"}\NormalTok{, }\AttributeTok{y =} \StringTok{"Maximum Daily Water Withdrawal (MGD)"}\NormalTok{, }\AttributeTok{color =} \StringTok{"Water System"}\NormalTok{)}

\NormalTok{water\_plot}
\end{Highlighting}
\end{Shaded}

\includegraphics{Kinser_A09_DataScraping_files/figure-latex/fetch.and.plot.Asheville.2015.data-1.pdf}

\begin{enumerate}
\def\labelenumi{\arabic{enumi}.}
\setcounter{enumi}{8}
\tightlist
\item
  Use the code \& function you created above to plot Asheville's max
  daily withdrawal by months for the years 2010 thru 2019.Add a smoothed
  line to the plot.
\end{enumerate}

\begin{Shaded}
\begin{Highlighting}[]
\CommentTok{\#9}
\NormalTok{Asheville\_2010 }\OtherTok{\textless{}{-}} \FunctionTok{scrape.it}\NormalTok{(}\StringTok{"01{-}11{-}010"}\NormalTok{, }\DecValTok{2010}\NormalTok{)}
\NormalTok{Asheville\_2011 }\OtherTok{\textless{}{-}} \FunctionTok{scrape.it}\NormalTok{(}\StringTok{"01{-}11{-}010"}\NormalTok{, }\DecValTok{2011}\NormalTok{)}
\NormalTok{Asheville\_2012 }\OtherTok{\textless{}{-}} \FunctionTok{scrape.it}\NormalTok{(}\StringTok{"01{-}11{-}010"}\NormalTok{, }\DecValTok{2012}\NormalTok{)}
\NormalTok{Asheville\_2013 }\OtherTok{\textless{}{-}} \FunctionTok{scrape.it}\NormalTok{(}\StringTok{"01{-}11{-}010"}\NormalTok{, }\DecValTok{2013}\NormalTok{)}
\NormalTok{Asheville\_2014 }\OtherTok{\textless{}{-}} \FunctionTok{scrape.it}\NormalTok{(}\StringTok{"01{-}11{-}010"}\NormalTok{, }\DecValTok{2014}\NormalTok{)}
\NormalTok{Asheville\_2015 }\OtherTok{\textless{}{-}} \FunctionTok{scrape.it}\NormalTok{(}\StringTok{"01{-}11{-}010"}\NormalTok{, }\DecValTok{2015}\NormalTok{)}
\NormalTok{Asheville\_2016 }\OtherTok{\textless{}{-}} \FunctionTok{scrape.it}\NormalTok{(}\StringTok{"01{-}11{-}010"}\NormalTok{, }\DecValTok{2016}\NormalTok{)}
\NormalTok{Asheville\_2017 }\OtherTok{\textless{}{-}} \FunctionTok{scrape.it}\NormalTok{(}\StringTok{"01{-}11{-}010"}\NormalTok{, }\DecValTok{2017}\NormalTok{)}
\NormalTok{Asheville\_2018 }\OtherTok{\textless{}{-}} \FunctionTok{scrape.it}\NormalTok{(}\StringTok{"01{-}11{-}010"}\NormalTok{, }\DecValTok{2018}\NormalTok{)}
\NormalTok{Asheville\_2019 }\OtherTok{\textless{}{-}} \FunctionTok{scrape.it}\NormalTok{(}\StringTok{"01{-}11{-}010"}\NormalTok{, }\DecValTok{2019}\NormalTok{)}

\NormalTok{Asheville\_water }\OtherTok{\textless{}{-}} \FunctionTok{rbind}\NormalTok{(Asheville\_2010, Asheville\_2011, Asheville\_2012, Asheville\_2013, Asheville\_2014, Asheville\_2015, Asheville\_2016, Asheville\_2017, Asheville\_2018, Asheville\_2019)}

\NormalTok{Asheville\_plot }\OtherTok{\textless{}{-}} \FunctionTok{ggplot}\NormalTok{(Asheville\_water, }\FunctionTok{aes}\NormalTok{(}\AttributeTok{x =}\NormalTok{ Date, }\AttributeTok{y =}\NormalTok{ Max\_Withdrawals)) }\SpecialCharTok{+}
  \FunctionTok{geom\_point}\NormalTok{()}\SpecialCharTok{+}
  \FunctionTok{geom\_smooth}\NormalTok{(}\AttributeTok{method =}\NormalTok{ lm, }\AttributeTok{se =} \ConstantTok{FALSE}\NormalTok{)}\SpecialCharTok{+}
  \FunctionTok{labs}\NormalTok{(}\AttributeTok{title =} \StringTok{"Asheville Maximum Water Withdrawals"}\NormalTok{, }\AttributeTok{x =} \StringTok{"Date"}\NormalTok{, }\AttributeTok{y =} \StringTok{"Maximum Daily Water Withdrawal (MGD)"}\NormalTok{)}

\NormalTok{Asheville\_plot}
\end{Highlighting}
\end{Shaded}

\includegraphics{Kinser_A09_DataScraping_files/figure-latex/unnamed-chunk-2-1.pdf}

\begin{quote}
Question: Just by looking at the plot (i.e.~not running statistics),
does Asheville have a trend in water usage over time? The plot suggests
that there has been an increase in maximum water usage from 2010 through
2019.
\end{quote}

\end{document}
